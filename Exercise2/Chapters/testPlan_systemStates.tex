
\section{System States}
	\begin{enumerate}
		\item {\bf ACC system in off state}
			\begin{enumerate}[label*=\arabic*.]
				\item Off state can be set from stand-by state or active state, when 
				the ACC system is turn off, by manually or/and automatically after self test.
				\item Off state can be set from stand-by state or active state automatically
				forced by a failure reaction.
			\end{enumerate}
		\item {\bf ACC system in stand-by state}
			\begin{enumerate}[label*=\arabic*.]
				\item Stand-by state can be reached from off state, manually or/and 
				automatically after self test.
				\item Stand-by state can be reached from active state through a human 
				interface
					\begin{enumerate}[label*=\arabic*.]
						\item Breaking by the driver shall deactivate ACC function at least 
						if the driver initiated brake force demand is higher than the ACC 
						initiated brake force.
						\item For type 1a and 2a systems, the stand-by state may be reached
						from active state when the driver depresses the clutch pedal.
					\end{enumerate}
			\end{enumerate}
		\item {\bf ACC system in active state}
			\begin{enumerate}[label*=\arabic*.]
				\item The active state can only be reached from stand-by state via the human
				interface. 
				\item The active state should switch between time-gap and set speed control
				according to the goal “Control basis”.
			\end{enumerate}
	\end{enumerate}

\clearpage

	\begin{table}[H]
		\begin{tabular}{| p{4cm} | p{10cm} |}
			\hline
			\rowcolor{gray}
			{\bf Test ID} & 2.1 \\ \hline
			{\bf Test name} & ACC\_off\_TEST\_manual \\ \hline
			{\bf System to be tested} & ACC1.1\\ \hline
			{\bf Main goal} & Maintain[System states] \\ \hline
			{\bf Goal} & Achieve [off state When ACC is turned off]\\ \hline
			{\bf Requirement(s)} to be tested & 
			{\bf 1.1} Off state can be set from stand-by state or active state, when 
			the ACC system is turned off, manually \\ \hline
			{\bf Description} & Test if the ACC system changes when the ACC system is 
			turned off, manually \\ \hline
			{\bf Stakeholders} & Tester, Developer \\ \hline
			{\bf Test type} & Unit test, System test \\ \hline
			{\bf Pre-conditions} & \\ \hline
			{\bf Input} & The user turns the ACC system off, via an interface \\ \hline
			{\bf Success criteria(s)} & ACC is set to OFF state. \\ \hline
			{\bf Stop critera(s)} & ACC is not set to OFF state. \\ \hline
			{\bf Dependencies} & none \\ \hline
		\end{tabular}
	\end{table}

	\begin{table}[H]
		\begin{tabular}{| p{4cm} | p{10cm} |}
			\hline
			\rowcolor{gray}
			{\bf Test ID} & 2.2 \\ \hline
			{\bf Test name} & ACC\_off\_TEST\_auto\\ \hline
			{\bf System to be tested} & ACC1.1\\ \hline
			{\bf Main goal} & Maintain[System states] \\ \hline
			{\bf Goal} & Achieve [off state When ACC is turned off ]\\ \hline
			{\bf Requirement(s)} to be tested & 
			{\bf 1.1} Off state can be set from stand-by state or active state, when the 
			ACC system is turned off, automatically after self test.\\ \hline
			{\bf Description} & Test if the ACC system state changes to off state when 
			the ACC system is turned off, automatically after self test \\ \hline
			{\bf Stakeholders} & Tester, Developer \\ \hline
			{\bf Test type} & Unit test, System test \\ \hline
			{\bf Pre-conditions} & \\ \hline
			{\bf Input} & An automatic self test that implies that the ACC should be turned 
			off \\ \hline
			{\bf Success criteria(s)} & ACC is set to OFF state.\\ \hline
			{\bf Stop critera(s)} & ACC is not set to OFF state. \\ \hline
			{\bf Dependencies} & none\\ \hline
		\end{tabular}
	\end{table}

	\begin{table}[H]
		\begin{tabular}{| p{4cm} | p{10cm} |}
			\hline
			\rowcolor{gray}
			{\bf Test ID} & 2.3 \\ \hline
			{\bf Test name} & ACC\_off\_TEST\_fail\\ \hline
			{\bf System to be tested} & ACC1.1\\ \hline
			{\bf Main goal} & Maintain[System states]\\ \hline
			{\bf Goal} & Achieve [off state By fail recognition]\\ \hline
			{\bf Requirement(s)} to be tested & {\bf 1.2} Off state can be set 
			from stand-by state or active state automatically forced by a failure 
			reaction. \\ \hline
			{\bf Description} & Test if the ACC is turned OFF if serious errors occur 
			\\ \hline
			{\bf Stakeholders} & Tester, Developer \\ \hline
			{\bf Test type} &  Unit test, System test \\ \hline
			{\bf Pre-conditions} & The system must be set to OFF from both 
			ACC\_ON and ACC\_STAND\_BY state. Must be done in different operations.\\ \hline
			{\bf Input} & Serious system error \\ \hline
			{\bf Success criteria(s)} & ACC is set to OFF state.\\ \hline
			{\bf Stop critera(s)} & ACC is not set to OFF state.\\ \hline
			{\bf Dependencies} & none \\ \hline
		\end{tabular}
	\end{table}

	\begin{table}[H]
		\begin{tabular}{| p{4cm} | p{10cm} |}
			\hline
			\rowcolor{gray}
			{\bf Test ID} & 2.4 \\ \hline
			{\bf Test name} & ACC\_standBy\_TEST\_manually \\ \hline
			{\bf System to be tested} & ACC1.1\\ \hline
			{\bf Main goal} & Maintain[System states] \\ \hline
			{\bf Goal} & Achieve[stand-by state from off state]\\ \hline
			{\bf Requirement(s)} to be tested & {\bf 2.1} Stand-by state can be reached 
			from off state, manually.\\ \hline
			{\bf Description} & Test if the ACC system state changes to off state when 
			the ACC system is turned off, manually\\ \hline
			{\bf Stakeholders} & Tester, Developer \\ \hline
			{\bf Test type} & Unit test, System test \\ \hline
			{\bf Pre-conditions} & The ACC system is in off state \\ \hline
			{\bf Input} & The user turns the changes the ACC system to stand-by state, 
			via an interface \\ \hline
			{\bf Success criteria(s)} & ACC is set to stand-by state. \\ \hline
			{\bf Stop critera(s)} & ACC is not set to stand-by state.\\ \hline
			{\bf Dependencies} & none \\ \hline
		\end{tabular}
	\end{table}

	\begin{table}[H]
		\begin{tabular}{| p{4cm} | p{10cm} |}
			\hline
			\rowcolor{gray}
			{\bf Test ID} & 2.5 \\ \hline
			{\bf Test name} & ACC\_standBy\_TEST\_automatically\\ \hline
			{\bf System to be tested} & ACC1.1\\ \hline
			{\bf Main goal} & Maintain[System states] \\ \hline
			{\bf Goal} & Achieve [stand-by state from off state] \\ \hline
			{\bf Requirement(s)} to be tested & {\bf 2.1} Stand-by state can be reached 
			from off state, automatically after self test.
			\\ \hline
			{\bf Description} & Test if the ACC system state changes to off state when 
			the ACC system is turned off, automatically after self test. \\ \hline
			{\bf Stakeholders} & Tester, Developer \\ \hline
			{\bf Test type} & Unit test, System test \\ \hline
			{\bf Pre-conditions} & The ACC system is in off state \\ \hline
			{\bf Input} & An automatic self test that implies that the ACC should be set 
			in stand-by state \\ \hline
			{\bf Success criteria(s)} & ACC is set to stand-by state.\\ \hline
			{\bf Stop critera(s)} & ACC is not set to stand-by state. \\ \hline
			{\bf Dependencies} & none \\ \hline
		\end{tabular}
	\end{table}

	

	\begin{table}[H]
		\begin{tabular}{| p{4cm} | p{10cm} |}
			\hline
			\rowcolor{gray}
			{\bf Test ID} & 2.6 \\ \hline
			{\bf Test name} & ACC\_standbyState\_viaBreaking \\ \hline
			{\bf System to be tested} & ACC1.1\\ \hline
			{\bf Main goal} & Maintain[System states] \\ \hline
			{\bf Goal} & Achieve [Deactivate when ACC by breaking] \\ \hline
			{\bf Requirement(s)} to be tested & {\bf 2.2.1} Breaking by the driver shall 
			deactivate ACC function at least if the driver initiated brake force demand 
			is higher than the ACC initiated brake force.\\ \hline
			{\bf Description} & Test if the ACC will turn off automatically if the driver 
			brakes harder than the applied brake force initiated by the ACC. The driver 
			must be in control if applying more force than the ACC. \\ \hline
			{\bf Stakeholders} & Developer\\ \hline
			{\bf Test type} & Unit test, System test \\ \hline
			{\bf Pre-conditions} & Car is in motion and driving at speed greater than 10km/h.
			ACC applies brake force.\\ \hline
			{\bf Input} & Breaking by the user, with greater force than the ACC \\ \hline
			{\bf Success criteria(s)} & ACC is set to stand-by state.\\ \hline
			{\bf Stop critera(s)} & ACC is not set to stand-by state.\\ \hline
			{\bf Dependencies} & none \\ \hline
		\end{tabular}
	\end{table}

	\begin{table}[H]
		\begin{tabular}{| p{4cm} | p{10cm} |}
			\hline
			\rowcolor{gray}
			{\bf Test ID} & 2.7 \\ \hline
			{\bf Test name} & ACC\_standbyState\_viaClutching \\ \hline
			{\bf System to be tested} & ACC1.1\\ \hline
			{\bf Main goal} & Maintain[System states] \\ \hline
			{\bf Goal} & Achieve [Deactivate ACC Driver press clutch] \\ \hline
			{\bf Requirement(s)} to be tested & {\bf 2.2.2} For type 1a and 2a systems, 
			the stand-by state may be reached from active state when the driver depresses 
			the clutch pedal.\\ \hline
			{\bf Description} & Test if the system reaches the stand-by state via clutching
			\\ \hline
			{\bf Stakeholders} & Developer \\ \hline
			{\bf Test type} & Unit tests \\ \hline
			{\bf Pre-conditions} & The car is a type 1a or 2a, car is in motion, driving 
			at speed greater than 10km/h.\\ \hline
			{\bf Input} & User presses the clutch pedal\\ \hline
			{\bf Success criteria(s)} & ACC is set to stand-by state. \\ \hline
			{\bf Stop critera(s)} & ACC is not set to stand-by state.\\ \hline
			{\bf Dependencies} & none \\ \hline
		\end{tabular}
	\end{table}

\begin{table}[H]
		\begin{tabular}{| p{4cm} | p{10cm} |}
			\hline
			\rowcolor{gray}
			{\bf Test ID} & 2.8 \\ \hline
			{\bf Test name} & ACC\_activeState\_viaInterface\\ \hline
			{\bf System to be tested} & ACC1.1\\ \hline
			{\bf Main goal} & Maintain[System states]\\ \hline
			{\bf Goal} & Achieve [active state  from human interface] \\ \hline
			{\bf Requirement(s)} to be tested & {\bf 3.1} The active state can only be 
			reached from stand-by state via the human interface. \\ \hline
			{\bf Description} & Test if the ACC system state changes from stand-by state 
			to active-state when the user interacts via the user interface, that implies 
			that the system should be set to active-state. \\ \hline
			{\bf Stakeholders} & Tester, Developer \\ \hline
			{\bf Test type} & Unit test, System test \\ \hline
			{\bf Pre-conditions} & The ACC system is in stand-by state \\ \hline
			{\bf Input} & The user interacts with the user interface, to set the ACC to 
			active-state \\ \hline
			{\bf Success criteria(s)} & ACC is set to stand-by state. \\ \hline
			{\bf Stop critera(s)} & ACC is not set to stand-by state. \\ \hline
			{\bf Dependencies} & none \\ \hline
		\end{tabular}
	\end{table}

	\begin{table}[H]
		\begin{tabular}{| p{4cm} | p{10cm} |}
			\hline
			\rowcolor{gray}
			{\bf Test ID} & 2.9 \\ \hline
			{\bf Test name} & ACC\_activeState\_ControlBasis\\ \hline
			{\bf System to be tested} & ACC1.1\\ \hline
			{\bf Main goal} & Maintain[System states]\\ \hline
			{\bf Goal} & Achieve [state according to "Control basis"]\\ \hline
			{\bf Requirement(s)} to be tested & {\bf 3.2} The active state should switch 
			between time-gap and set speed control according to the goal “Control basis”.
			\\ \hline
			{\bf Description} & Test if the active-state switches between time-gap and 
			set-speed control, according to the “Control basis” goal. \\ \hline
			{\bf Stakeholders} & Tester, Developer \\ \hline
			{\bf Test type} & Unit test, System test\\ \hline
			{\bf Pre-conditions} & \begin{itemize} 
			\item The ACC system is in active-state
			\item The car in front is in reach of the distance sensor
			\item Both cars are in motions 
			\end{itemize} \\ \hline
			{\bf Input} & measuredSpeed, measuredDistance \\ \hline
			{\bf Success criteria(s)} & The inner states in active-state are set according 
			to the “Control basis” goal\\ \hline
			{\bf Stop critera(s)} & The inner states in active-state aren’t set according 
			to the “Control basis” goal\\ \hline
			{\bf Dependencies} & none \\ \hline
		\end{tabular}
	\end{table}

	