\chapter{Test Strategy}
	
	{\Huge Introduction}

\clearpage
	\section{Stakeholders}

		{\bf Developers:} Will develop the system, and want the system to satisfy 
		the requirements from the customer. 

		{\bf End-user:} Will be using the ACC system in their everyday life, and therefore 
		wants a bug-free system, thus the system operates according to the system. 
		They also want a high security/safety margin.

		{\bf Testers:} Will test the system. The testers do also need to assure a 
		specific testing coverage so the QA department can assure the quality level 
		of the ACC system to the customer. 

		{\bf Customer:} The owner of the ACC system, wants the system to operate 
		according to the requirements, low cost and short development time.


	\section{Test Types}

		{\bf Acceptance testing:} Is the final level of testing were a test is conducted to determine 
		if the requirements of a specification or contract are met. This test is often held by the customer. 

		{\bf System testing:} is done by testing a complete, integrated system to evaluate the system's
		compliance with its specified requirements.
		
		{\bf Component testing:} Is taking a component of a system and the testing of the component 
		is done in isolation from the rest of the system.
		
		{\bf Integration testing:} individual software modules are combined and tested as a group. 
		This is often done on modules/components that are dependent on each other.
		
		{\bf Unit testing:} is testing individual units of code (often one unit test for each method).
		Often testing that input gives the right output.
	
	\clearpage
	\section{Test Levels}
				
		{\bf Security:} In this level will we look at and test the security of the system. 
		It’s important that the system is safe to use, so we can avoid accidents on humans, 
		other vehicles, environment etc. This makes this level the most important one in this 
		test strategy. Tests that will help testing the security is Acceptance tests and system 
		tests, thus black box testing. Stakeholders in this test level will be customer and 
		testers.

		{\bf Logical:} The logical test level is focusing on the logical aspects in the system. 
		This is done by using components test and unit test, so this will be white box and grey 
		box testing. To test the logical level of the system is important to make sure that the 
		security is good on a low level. Stakeholders that is important in this test level is 
		developers and testers.

		{\bf Signal:} The signal test level is focusing on testing that the different components 
		that sends or receive signals is working satisfactory. Examples is the component measuring 
		the distance to the car in front,  the component that makes sure the speed isn’t higher 
		than the setSpeed, etc. This is also a test level that will make sure that the security of 
		the system is high. To test this we will use components tests and unit tests. We will use 
		grey box testing for the components and white box testing for the unit testing. Stakeholders 
		for this test level would be tester.
		
		{\bf Hardware:} On this test level is the focus testing that all the hardware is OK 
 		and has no defects. This is the lowest level of the testing levels. Because the first 
 		thing we have to test before we can do the other test is testing that all the hardware 				
 		is running. The test type we will use here is component testing, and since the components 
 		comes from external suppliers, this will typically be done as a black box test. 
 		Stakeholders for this test level is tester. 
 					
	
	\clearpage
	\section{Test environment}	

		{\bf Security:} The testing environment at a security level needs to be as realistic 
		as possible. At this test level we need at least two cars, a driving lane with two tracks 
		and a speedometer to track the physical speed of the cars. For the testing we also need
		an experienced test driver. 

		{\bf Logical:} At a logical level we only need a simulation of the ACC system, only a pc 
		is required.
				
		{\bf Signal:} At this level, we need to have a simulation of the ACC system as well as all
		the sensors and controllers.

		{\bf Hardware:} For this testing level we need all the hardware of the ACC system.

	\section{Documentation}

		\begin{itemize}
			\item {\bf Security:} ISO standard for security and ACC system
			\item {\bf Logical:} Input type required from signal level and 
			description of all signals
			\item {\bf Signal:} Format type for each signal type and a mapping 
			from signal to hardware.
 			\item {\bf Hardware:} Components used in each car type, component 
 			suppliers and all Hardware spesificatons.
 		\end{itemize}

	\clearpage
	\section{Completion Criteria}

		\begin{table}[H]
			\begin{tabular}{ p{2cm} | p{11cm} }
				\hline
				{\bf Test Level} & {\bf Completion Criteria} \\ \hline
				{\bf Security} & 
					\begin{itemize}
						\item The ACC system is in the correct state
 						\item The physical speed is correct
						\item The physical distance to other car is correct 
						\item The ACC system should not cause any damage to persons or physical objects.
					\end{itemize}
				\\ \hline

				{\bf Logical} & 
					\begin{itemize}
						\item A specified input to a logical component should give the right output.
						\item Measured speed equals physical speed 
						\item Measured distance equals physical distance
					\end{itemize}
				\\ \hline

				{\bf Signal} &
				 \begin{itemize}
						\item A signal should be sent to the correct component
						\item A signal should be recieved from the correct component
						\item A signal should be recived/sent with a minimum delay
						\item Signals are formatted and sent to the logical level with an appropriate format.
					\end{itemize}
				\\ \hline

 				{\bf Hardware} & 
 					\begin{itemize}
						\item No defects are detected at the hardware level
					\end{itemize}
 				\\ \hline
			\end{tabular}
		\end{table}
