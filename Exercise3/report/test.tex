\chapter{Test Result}

	Lorem ipsum dolor sit amet, consectetur adipisicing elit, sed do eiusmod
	tempor incididunt ut labore et dolore magna aliqua. Ut enim ad minim veniam,
	quis nostrud exercitation ullamco laboris nisi ut aliquip ex ea commodo
	consequat. Duis aute irure dolor in reprehenderit in voluptate velit esse
	cillum dolore eu fugiat nulla pariatur. Excepteur sint occaecat cupidatat non
	proident, sunt in culpa qui officia deserunt mollit anim id est laborum.

	\begin{center}
		
	    \begin{longtable}{| l | l | l | p{3cm}  | l | p{5cm} |}
	    \hline
	    \rowcolor{gray}
	    ID & When & Tester & Expected result & Accepted & Comment \\ \hline
	    1.1 & 14.4.3 & Odd & The invalid SWComponent is not added to the system & No & When trying to add a new component with invalid input, for instance a string in the version or subversion field, the system crashes.\\ \hline
	    1.2 & 14.4.3 & Odd & The already existing SWComponent is not added to the system & Yes & The already existing SWComponent is actually added, but the system automatically increments the subversion number to a number that is not in the archive. Therefore the added component becomes unique. \\ \hline
	    1.3 & 14.4.3 & Odd & A new software component is added to the system & Yes &  New SW component were saved in the archive.\\ \hline
	    2.1 & 14.4.3 & Odd & A new, valid, action script is uploaded to the Action database & Yes/No &  Managed to bind a SWComponent to a new ECU/but can’t update an already existing ECU to a another SWComponent\\ \hline
	    2.2 & 14.4.3 & Odd & The invalid actionscript is not uploaded to the Action database & No & When trying to add a new actionscript with invalid input, for instance a string in the Software ID field, the system crashes. \\ \hline
	    4 & 14.4.3 & Odd & The list of cars is sent to the garage & No & The Factory part of the system says that the list is sent, but it is impossible to know if the garage has received it.\\ \hline
	    5 & 14.4.3 & Odd & To be able to find registred owners & Yes &  Was able to search for register owners\\ \hline
	    6 & 14.4.3 & Odd & To be able to send email to the owner of the spesific car & No & When trying to send email to owner, the system displays a message with the address (not email address) of the owner as text… Tried to register owner with valid email, email was not received. \\ \hline
	    7 & 14.4.3 & Linn & A new series of vehicle is added to the database & Yes & There doesn't exist any functionality to add a new series of vehicle \\ \hline
	    8 & 14.4.3 & Odd & A new vehicle is added to a production series & Yes & No problem to add a new vehicle to the DB. The ID must be one higher than the former highest ID. \\ \hline
	    X & DATE & TESTER & MORE TESTS & Yes &  \\ \hline

	    \hline
	    \end{longtable}
	\end{center}

	\section{Test interpretation}

		It seems that the system have some functionality, but it’s still a lot to work to be done before it is a stable and usable system. Some of the missing functionality is caused by minor software bugs as stated in the previous section.

		The SoCam system can be divided into two parts, the fabrication- and the garage part. From what we can understand it does not seem like the two parts are connected to each other, at least they do not communicate on any level.  The system gives no feedback on any actions, which makes it difficult to know if you are connected to the system. 

		Based on the current version of the system, it is not applicable for its given purpose. The system does not operate to a satisfiable level given by the criterias. Therefor we can not recommend using the system before the system errors has been fixed. The discovered errors and bugs has been reported. 

	\section{Error report}

		The following list of function states the failed functional tests. The list is prioritized based on what level of severity. We have rated the severity of the errors based on the errors or malfunctions occuring and how often the tasks are performed. 

		The failed tests with highest priority should also be the ones fixed first. An error is defined as a test that did not pass the given criteria. In the table above, these tests are marked with “No” in the column named “Accepted”. 

		Priority and Test ID
		\begin{enumerate}
			\item 1.1 - Add a new software component
			\item 4 - Send a list of cars to the garage
			\item 7 - A new series of vehicle is added to the database
			\item 6 - Send an email to the owner of the car
			\item 2.1 - Add a new, valid, action script to the Action database
			\item 2.2 - A new, not valid, action script to the Action database

		\end{enumerate}	


	\section{Conclusion}

		This chapter will give concluding remarks on our test plan, tests performed, risk assessment and test evaluation. 

		After having planned and performed testing and risk assessment for the system SoCam, we have discovered several errors and malfunctions that must be fixed before the system may be sold to a customer. The system has errors with roots deep in the software code, and the developer should consider starting from scratch. Some of the errors discovered may be fixed by software patches. 

		The risk assessment shows that there are errors that may cause dangerous situations when using SoCam. Among these aspects there is a high potential for improvement. The system consists of many critical components which rely on each other, which leads to potential danger if one component fails. The system also relies heavily on its databases, which may lead to system failure and errors if the communication with the database is not functioning properly. SoCam’s user interface is difficult to use, which may lead to increased rate of system failures. 

		These factors collectively result in a situation where we do not recommend using SoCam in its current state.


