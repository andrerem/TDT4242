\chapter{Test Strategy}


	The test strategy is important in order to obtain consensus on test goals and objectives from stakeholders – e.g. management, developers, testers, customers and users It is also important to manage expectations right from the start and be sure that the product are heading in the right direction.

	This section will give a description of what's being testet, stakeholders, test types and test leves with focus on test environment, documentation and completion criteria.

	\clearpage

	\section{What’s being tested?}

		In this system we have chosen to do  an unit test, 
		an integration test and a system test. 
		These tests covers almost the whole area of the system, especially data, without adding 
		or changing the codebase.

		We’ll derive a short strategy for unit testing level and integration testing level, further we’ll make a more detailed plan and strategy for the system testing level. The plan for system testing will include a test set. The test set is derived from the use cases and their flow of events, and they are sorted by their total risk derived in the risk assessment chapter. The tests will be executed, and the result will be the basis for our conclusion whether to ship or not to ship the system.

	\section{Stakeholders}

		{\bf Developers:} Will develop the system, and want the system to satisfy the requirements from the customer. Will in this case do the risk assessment and testing before handing over the system to the testers.

		{\bf End-user:} Will be using the SoCam system in their everyday life, and therefore wants a bug-free system, thus the system operates according to the requirements. They also want a high security/safety margin.

		{\bf Testers:} Will test the system. The testers do also need to assure a specific testing coverage so the QA department can assure the quality level of the SoCam system to the customer.   

		{\bf Customer:} The owner of the SoCam system, wants the system to operate according to the requirements, low cost and short development time.


	\section {Test Types}

		\begin{itemize}
			\item {\bf System testing:} is done by testing a complete, integrated system to 
			evaluate the system's compliance with its specified requirements. We don’t need 
			any knowledge og the source code or the logic behind it.
			\item {\bf Integration testing:} individual software modules are combined and tested as a group. 
			This is often done on modules/components that are dependent on each other. 
			\item {\bf Unit testing:} is testing individual units of code (often one unit test for each method). 
			Often testing that input gives the right output. 
		\end{itemize}

	When performing the tests we have based our test plan on the test sets given below. The test sets covers system test, integration tests and unit tests. We plan on performing the system and integration tests using the black box technique. For the unit tests we plan on using a white box technique. 

	\section{Criteria for test completion}

		\begin{itemize}
			\item {\bf System testing:} For a successful system test every input should give the equally right 
			output, and it should be shown in the user interface of the system. 
			\item {\bf Integration testing:} For a successful integration test the different modules in the 
			system are combined into suitable groups and tested for every input that gives the right output.  
			\item {\bf Unit testing:} The test is successful if the different units of the system gives the 
			right output on the different inputs that are given. 
		\end{itemize}

	\section {Test environment}

	To be able to test the SoCam system there is no need in fancy equipment or room. The only thing we need 
	is a PC with the necessary software installed and a test user to test the program with.



	