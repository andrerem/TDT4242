\chapter{Test Result}

	This chapter includes all executed tests from the test plan from chapter 5, as well
	as the test interpretation, error report, and conclusion. 
	
	The executed tests will give a summary of tests that are executed associated with
	attributes describing expected results, an acceptance evaluation and a comment. 
	All the results are interpreted in the next section, and the results will be used
	in the conclusion where we will give a recommendation wheter to ship og not ship
	the system based on the intepretation of the test results. 

	\clearpage

	\section{Executed Tests}
	\begin{center}
		
	    \begin{longtable}{| l | l | l | p{3cm}  | l | p{5cm} |}
	    \hline
	    \rowcolor{gray}
	    ID & When & Tester & Expected result & Accepted & Comment \\ \hline
	    
	    1 
	    & 14.4.3 
	    & Odd 
	    & The invalid SWComponent is not added to the system 
	    & No 
	    & When trying to add a new component with invalid input, for instance a string in 
	    the version or subversion field, the system crashes.\\ \hline
	    
	    2 
	    & 14.4.3 
	    & Odd 
	    & The already existing SWComponent is not added to the system 
	    & Yes 
	    & The already existing SWComponent is actually added, but the system automatically 
	    increments the subversion number to a number that is not in the archive. Therefore the 
	    added component becomes unique. \\ \hline
	    
	    3 
	    & 14.4.3 
	    & Odd 
	    & A new software component is added to the system 
	    & Yes 
	    &  New SW component were saved in the archive.\\ \hline

	    4 
	    & 14.4.3 
	    & Odd 
	    & A new, valid, action script is uploaded to the Action database 
	    & No 
	    &  Managed to bind a SWComponent to a new ECU/but can’t update an already existing 
	    ECU to a another SWComponent\\ \hline

	    5 
	    & 14.4.3 
	    & Odd 
	    & The invalid actionscript is not uploaded to the Action database & No & When trying 
	    to add a new actionscript with invalid input, for instance a string in the Software 
	    ID field, the system crashes. \\ \hline

	    6 
	    & 14.4.3 
	    & Odd & The list of cars is sent to the garage 
	    & No 
	    & The Factory part of the system says that the list is sent, but it is impossible 
	    to know if the garage has received it.\\ \hline

	    7 
	    & 14.4.3 
	    & Odd 
	    & To be able to find registred owners 
	    & Yes 
	    & Was able to search for register owners\\ \hline

	    8 
	    & 14.4.3 
	    & Odd 
	    & To be able to send email to the owner of the spesific car 
	    & No 
	    & When trying to send email to owner, the system displays a message with the address 
	    (not email address) of the owner as text… Tried to register owner with valid email, 
	    email was not received. \\ \hline

	    9 
	    & 14.4.3 
	    & Linn 
	    & A new series of vehicle is added to the database 
	    & No & There doesn't exist any functionality to add a new series of vehicle \\ \hline

	    10 
	    & 14.4.3 
	    & Odd 
	    & A new vehicle is added to a production series 
	    & Yes 
	    & No problem to add a new vehicle to the DB. The ID must be one higher than the 
	    former highest ID. \\ \hline

	    11 
	    & 14.4.16 
	    & Marte 
	    & Find the vehicles serial number by searching on the owners name 
	    & Yes 
	    & By searching on the owners name we found the serial number of the vehicle. By searching 
	    on a non registered name we got a message that the customer is not in the database and 
	    no serial number is found. \\ \hline

	    12 
	    & 14.04.16 
	    & Marte 
	    & Download the latest software update to the local system 
	    & No 
	    & There is no functionality for downloading the latest software updates. \\ \hline

	    13 
	    & 14.04.16 
	    & Marte 
	    & Update the vehicles configuration info 
	    & No 
	    & Did not manage to connect to the cental system so we the vehicles configuration 
	    info was not updated. \\ \hline 

	    \hline
	    \end{longtable}
	\end{center}

	\section{Test Interpretation}

		It seems that the system has some functionality, but it’s still a lot to work to be 
		done before it is a stable and usable system. Some of the missing functionality is 
		caused by minor software bugs as stated in the previous section.

		The SoCam system can be divided into two parts, the fabrication- and the garage part. 
		From what we can understand it does not seem like the two parts are connected to each 
		other, at least they do not communicate on any level.  The system gives no feedback on 
		any actions, which makes it difficult to know if you are connected to the system. 

		Based on the current version of the system, it is not applicable for its given purpose. 
		The system does not operate to a satisfiable level given by the criterias. Therefore we 
		can not recommend using the system before the system errors has been fixed. The discovered 
		errors and bugs has been reported. 

	\clearpage
	\section{Error report}

		The following list of function states the failed functional tests. The list is prioritized 
		based on what level of severity. We have rated the severity of the errors based on the 
		errors or malfunctions occuring and how often the tasks are performed. 

		The failed tests with highest priority should also be the ones fixed first. An error is 
		defined as a test that did not pass the given criteria. In the table above, these tests 
		are marked with “No” in the column named “Accepted”. 

		Priority and Test ID
		\begin{enumerate}
			\item 1 - Add a new software component
			\item 6 - Send a list of cars to the garage
			\item 9 - A new series of vehicle is added to the database
			\item 8 - Send an email to the owner of the car
			\item 13 - Update the vehicles configuration info
			\item 12 - Download the latest software update to the local system
			\item 4 - Add a new, valid, action script to the Action database
			\item 5 - A new, not valid, action script to the Action database
			
		\end{enumerate}	


	\section{Conclusion}

		This chapter will give concluding remarks on our test plan, tests performed, risk 
		assessment and test evaluation. 

		After having planned and performed testing and risk assessment for the system SoCam, 
		we discovered several errors and malfunctions that must be fixed before the system may 
		be sold to a customer. The system has errors with roots deep in the software code, and 
		the developers should consider starting from scratch. Some of the errors discovered may 
		be fixed by software patches. 

		The risk assessment shows that there are errors that may cause dangerous situations when 
		using SoCam. Among these aspects there is a high potential for improvement. The system 
		consists of many critical components which rely on each other, which leads to potential 
		danger if one component fails. The system also relies heavily on its databases, which may 
		lead to system failure and errors if the communication with the database is not functioning 
		properly. SoCam’s user interface is difficult to use, which may lead to increased rate of 
		system failures. 

		These factors collectively result in a situation where we do not recommend shipping SoCam in 
		its current state.


