\chapter{Oversikt fra foiler}
	
	\section{Foil 2.0 - 2.2: Introduction}
		\begin{itemize}
			\item Main theme:
				\begin{itemize}
					\item Requirements without tests will be ignored
					\item Tests without requirements are meaningless
				\end{itemize}
			\item What is a requirement?
			\item Challenges in requirement engineering
			\item Phenomena (world/shared/machine)
			\item Requirement statements
				\begin{itemize}
					\item Descriptive
					\item Predictive
				\end{itemize}
			\item Formulation of requirements
			\item Domain property
			\item Goal
				\begin{itemize}
					\item What is a goal?
					\item High-level Goal
					\item Requirement
					\item Assumption (expectation, responsability)
				\end{itemize}
			\item Goal statement typolology
			\item Goal categorization
			\item Functional vs non-functional goals
			\item Goal refinement tree
			\item Where do we get goals from?
			\item Qualitative goal-requirements tracing
			\item Qualitative metrics:
				\begin{itemize}
					\item Ambiguous, inconsistent, opaque, noisy, inncomplete, forward referencing
				\end{itemize}
			\item Ontology
		\end{itemize}

	
	\section{Foil 3.1 - Guided Natual Language (GNL) and Requirement Boilerplates }
		\begin{itemize}
			\item Level of requirements
				\begin{itemize}
					\item Informal
					\item Semiformal
					\item Formal
				\end{itemize}
			\item Requirement elicitation (step1 - step 4)
			\item Challenges with requirement elicitation
			\item Humans and machines
			\item GNL and BP
		\end{itemize}


	\section{Foil 3.2: Requirements Traceability}
		\begin{itemize}
			\item What is requirement traceability?
			\item Traceability goals (Validation, verification, system inspection and certification/audits)
			\item Challenges of traceability
			\item Traceability metamodels
			\item Approaches to traceability
				\begin{itemize}
					\item Manual trace links
					\item Scenario driven traceability	
					\item Trace by tagging
				\end{itemize}
			\item Footprints	
		\end{itemize}

	\section{Foil 3.3: Requirements Testability}

		\begin{itemize}
			\item Testability definition
			\item Testability concerns
				\begin{itemize}
					\item How easy is it to test the implementation?
					\item How test-friendly is the requirements?
				\end{itemize}
			\item Three ways to check that we have achived our goals:
				\begin{itemize}
					\item Executing tests
					\item Run Experiments
					\item Inspect code
				\end{itemize}
			\item When do we use what? (test, experiments, inspections)
			\item Testability challenges
			\item Making requirements testable
			\item Requirementss for testability (The "what" and the "why")
			\item A testable requirement need to be:	
				\begin{itemize}
					\item Correct
					\item Complete
					\item Consistent
					\item Clear
					\item Relevant
					\item Feasible
					\item Traceable
				\end{itemize}
			\item Some sound advice:
				\begin{itemize}
					\item Modifying Phrases
					\item Vague words
					\item Pronouns with no reference
					\item Passive voice
					\item Negative requirements
					\item Assumptions and comparisasions
				\end{itemize}
			\item Concerns releated to implementation and testability:
				\begin{itemize}
					\item Autonomy of the system under test
					\item Observability of the testing progress
					\item Re-test efficiency
					\item Test restartability
				\end{itemize}
		\end{itemize}


	\section{Foil 4.1: Advanced Use Cases}
		\begin{itemize}
			\item Advanced use cases vocabulary (Actor, use case, use case model)
			\item Finding actors
			\item Finding use cases
			\item Key points for use cases
			\item Reuse opportunity for use cases and relationships between use cases:
				\begin{itemize}
					\item Dependency
					\item Include
					\item Extends
					\item Generalize
				\end{itemize}
			\item Adding details
			\item From use case to sequence diagram
			\item Use case index
				\begin{itemize}
					\item Scope
					\item Complexity
					\item Status
					\item Priority
				\end{itemize}
			\item Use case diagrams: pros and cons
			\item Textual use case
				\begin{itemize}
					\item Use case number
					\item Application
					\item Use case name
					\item Use case description
					\item Primary actor
					\item Precondition
					\item Trigger
					\item Basic flow
					\item Alternative flow
				\end{itemize}
			\item Textual use case: pros and cons
			\item Mis-use cases
			\item Textual mis-use case
			\item Why mis-use case?
			\item Mis-use case: pros and cons
			\item Use case maps
			\item Use case Maps - path
		\end{itemize}


	\section{Foil 4.2: Test Driven Development (TDD)}
		\begin{itemize}
			\item Development and testing
			\item Why TDD
			\item Green field projects (the roots of TDD, start at "scratch"/clean slate)
			\item TDD operates with four pairs of strategies:
				\begin{itemize}
					\item What is a green field project?
					\item Details vs the "big picture"
					\item Uncertain territory vs. the familiar
					\item Highest value vs. "low-hanging fruits"
				\end{itemize}
			\item Essential TDD concepts:
				\begin{itemize}
					\item Fixtures
					\item Test doubles
					\item Guidelines for a testable design
					\item Unit test patterns 
						\begin{itemize}
							\item Assertion types: resulting state, guard, delta, custom, interaction
							\item Keep or throw away a unit test ?
						\end{itemize}
					\item Legacy code 
						\begin{itemize}
							\item Inflection points (test point)
							\item Test and change
						\end{itemize}
				\end{itemize}
			\item TDD Acceptance testing
				\begin{itemize}
					\item Pick a user story
					\item Write tests for the story
					\item Automate the tests
				\end{itemize}
		\end{itemize}

	\section{Foil 5.1: Test vs. Inspection - Part 1}

		\begin{itemize}
			\item Man vs. machine
			\item Types of inspection:
				\begin{itemize}
					\item Walkthrough process
					\item The informal inspection process
					\item The formal inspection process
						\begin{itemize}
							\item Planning
							\item Kickoff
							\item Improve the product
							\item Checking changes
							\item Individual checking
							\item Logging Meeting
						\end{itemize}
				\end{itemize}
			\item Testing process
			\item Testing types
				\begin{itemize}
					\item Unit testing
					\item Function verification testing
					\item System verification testing
				\end{itemize}
			\item Strong and week points
			\begin{itemize}
				\item Testing 
				\item Inspections
			\end{itemize}
		\end{itemize}

	\section{Foil 5.2: Test vs. Inspection - Part 2}

		\begin{itemize}
			\item Defect types
			\item Triggers
			\item Inspection
				\begin{itemize}
					\item Defect types
					\item Triggers
				\end{itemize}
			\item Testing
				\begin{itemize}
					\item Defect types
					\item Triggers
				\end{itemize}
			\item Inspection as a social process
				\begin{itemize}
					\item Gain and loss
					\item Nominal group vs. Real group
				\end{itemize}
		\end{itemize}


	\section{Testing and Cost/Benefit}
		\begin{itemize}
			\item Why cost/benefit?
			\item When should we stop testing?
			\item Which cost should be included?
			\item Which benefits should be included?
			\item Hard costs and "soft benefits"
				\begin{itemize}		
				 	\item Cost are now
				 	\item Benefits are often later 
				 \end{itemize}
			\item How to assign value to "soft benefits"?
			\item Creation of value
			\item Soft benefits
			\item Cost/benefits are used to decide when to stop testing:
				\begin{itemize}
					\item P(Wrong): the probability of making the wrong decision
					\item Cost(Wrong): the cost of making the wrong decision
				\end{itemize}
			\item Risk = P(wrong) * Cost(wrong)
			\item The value of information
			\item Regret - pay now or maybe later?
			\item Leverage, risk and regret: 
				\begin{itemize}
					\item Total benefit = regret + benefit
					\item Total cost = risk + cost
					\item Leverage = (total benefit - total cost)/total cost
				\end{itemize}
		\end{itemize}


	\section{Foil 6.1: Writing a Test Strategy}
		\begin{itemize}
			\item Strategy vs. plan
				\begin{itemize}	
					\item {\it A plan says "Here are the steps", while a strategy says
					"here are the best steps".} The strategy speaks to the {\bf reason why},
					while the plan is {\bf focused on how}.
				\end{itemize}	
			\item Why a testing strategy
			\item Testing strategy concepts
				\begin{itemize}
					\item Purpose of a test strategy
					\item Testing focus (Users, analysts, designer, programmer)
					\item Contents of a testing strategy
						\begin{itemize}
							\item Project plan, risks and activities
							\item Relevant regulations
							\item Required process and standards
							\item Supporting guidelines
							\item Stakeholders
							\item Necessary resources
							\item Test levels and phases
								\begin{itemize}
									\item Development requriements level
									\item Design level
									\item Implementation level
									\item Test level
								\end{itemize}
							\item Completion criteria for each phase
							\item Required documentation and review method for each document
						\end{itemize}
					\item Software integrity levels (IEE and ISO standards)
					\item Test objectives and priorities
					\item Test data selection
					\item Random testing
					\item Domain partition testing
					\item Risk based testing
					\item User profile testing
					\item Bach's risk-based testing
				\end{itemize}
		\end{itemize}


	\section{Foil 6.2: White Box and Black Box Testing}
		\begin{itemize}
			\item What is white box testing?
			\item Code coverage
			\item Path coverage
			\item Error messages
			\item What is black box testing?
			\item Testing real-time systems
			\item Basic scenario pattern (BSP)
			\item Key-event service pattern (KSP)
			\item Test Automation
			\item White box testing
				\begin{itemize}
					\item Static white box testing (Code inspection and walkthrough)
					\item Dynamic white box testing (Statement coverage, path coverage, decision coverage
					use coverage).
				\end{itemize}
		\end{itemize}


	\section{Foil 6.3: Grey Box Testing}
		\begin{itemize}
			\item 
		\end{itemize}

	\section{}

